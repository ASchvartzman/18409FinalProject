%Jennifer Pan, August 2011

\documentclass[11pt,letter]{article}
	% basic article document class
	% use percent signs to make comments to yourself -- they will not show up.
\usepackage{hyperref}
\usepackage{amsmath}
\usepackage{amssymb}
\usepackage{amsthm}
\usepackage{amsfonts}
\usepackage{mathtools}
\usepackage{complexity}
\usepackage{enumerate}
\usepackage[margin=1in]{geometry}
\newtheorem{definition}{Definition}
\newtheorem{theorem}{Theorem}
\newtheorem{lemma}{Lemma}

	% packages that allow mathematical formatting

\usepackage{graphicx}
	% package that allowxs you to include graphics

\usepackage{setspace}
	% package that allows you to change spacing

%\usepackage{fullpage}
	% package that specifies normal margins
	

\begin{document}
	% line of code telling latex that your document is beginning


\title{18.409 Final Project: More General Mixture Models}

\author{Wei Hu\\ \href{mailto:huwei@mit.edu}{huwei@mit.edu} \and Ariel Schvartzman \\ \href{mailto:arielsc@mit.edu}{arielsc@mit.edu}} 
% Note: when you omit this command, the current dateis automatically included
 
\maketitle 
	% tells latex to follow your header (e.g., title, author) commands.

\section{Introduction}

In this short paper we present techniques for learning a mixture of distributions for more general classes of distributions than the ones seen in class. We focus on two papers, their key theorems and how they are applied in algorithms with provable guarantees. 

The first paper, by Kannan et al. \cite{Kannan08spectral}, uses a spectral projection technique to learn a mixture of log-concave distributions efficiently provided that the means are separated. Their main theorem, which we present and prove, says that for an arbitrary mixture of distributions the SVD subspace of a sample is close to the means of the samples, where the closeness depends on the variances. Their algorithm works for a mixture of log-concave distributions, which generalizes previous results for Gaussians.

The second paper, by Dasgupta et al. \cite{Dasgupta05heavy-tail}, focuses on efficiently learning a mixture of symmetric heavy-tailed distributions whose expectation or variance might be infinite, with minimal separation requirements. They present an algorithm for learning when the centers of the distributions are known which uses the $\ell_1$ norm as a classifier, assuming some additional restrictions on the distributions. They also present an algorithm that works when the centers are not known and provably works for the family of distributions they consider. We will focus on the first result as it is easier to understand and provides some intuition on the second result. 

In both papers, the authors examine different ways to solve the following problem. There is a mixture of $k$ distributions $F_1, \cdots, F_k$ in $n$ dimensions, and the distribution $F_i$ has mixing weight $w_i$. A sample $x$ from the mixture is taken from distribution $F_i$ with probability $w_i$. The goal is to classify i.i.d. samples from $F$ and to approximately learn the underlying distributions.

\section{Spectral Projection Method}

%(use $n$ for dimension, $\varepsilon$ for $w_{min}$)

The method of spectral projection, i.e., projecting the samples onto the subspace spanned by the top $k$ singular vectors (the SVD subspace) of the distribution, was successfully applied to the special case of learning a mixture of spherical Gaussians, provided that the separation condition $|\mu_i - \mu_j| = (\sigma_i + \sigma_j) \Omega^*(k^{1/4})$ holds for any $i\not=j$ \cite{Vempala02aspectral}. This removed the dependence on $n$ in the separation condition. The key idea is that the SVD subspace of a mixture of spherical Gaussians contains the means of its $k$ components. However, this property is not valid for more general classes of distributions, e.g. non-spherical Gaussians.
Nevertheless, in \cite{Kannan08spectral} Kannan et al. show that spectral projection can approximately preserve the intermean distances ``on average". Based on this key result, they give an efficient algorithm for log-concave distributions, under the separation condition $|\mu_i - \mu_j| = (\sigma_i + \sigma_j) \Omega^*\left(\frac{k^{3/2}}{\varepsilon^2}\right)$, where $\varepsilon$ is a lower bound on the mixing weights $w_i$.

It is worth noting that this algorithm for log-concave distributions does not identify all the components after one spectral projection (as in the case of spherical Gaussians). Instead, it runs iteratively and identifies one component in each iteration through projection.
The reason is that one can only show that the intermean distances are preserved in an average sence after projection, and some component means may become very close so the samples from those components are indistinguishable. On the other hand, it is shown that there is one component that is still separated from others, and the algorithm correctly identifies such a component in each iteration.\\



\noindent\textbf{Notation.}
We use $|\cdot|$ to denote the $\ell_2$ distance in $\mathbb R^n$.
A mixture $F$ in $\mathbb R^n$ has $k$ components $F_1, \cdots, F_k$ with mixing weights $w_1, \cdots, w_k$. The mean of $F_i$ is $\mu_i$, and the maximum variance of $F_i$ in any direction is denoted by $\sigma_i^2$. For any subspace $W$, we denote the maximum variance of $F_i$ along any direction in $W$ by $\sigma_{i, W}^2$.
The orthogonal distance from a point $x$ to a subspace $W$ is denoted by $d(x, W)$.

For a set of i.i.d. samples $S$ from $F$, we partition $S$ as $S = S_1 \cup \cdots \cup S_k$, where $S_i$ is the set of samples from $F_i$. Let $\mu_i^S$ be the sample mean of $S_i$, i.e.,
$\mu_i^S = \frac{1}{|S_i|} \sum_{x\in S_i} x$.
For any subspace $W$, we denote the maximum sample variance of $S_i$ along any direction in $W$ by $\hat \sigma_{i, W}^2(S)$.

\subsection{Main theorem on spectral projection}

For a set of points $S$ in $\mathbb R^n$, let $A$ be the matrix whose rows are the points in $S$. Then the subspace spanned by the top $k$ right singular vectors of $A$ is called the \emph{SVD subspace} of $S$. The following theorem from \cite{Kannan08spectral} is an important property of spectral projection. We follow the notations introduced before.

\begin{theorem}\label{thm:spectral-svd}
Let $W$ be the SVD subspace of $S$. Then
\[
\sum_{i=1}^k |S_i| d(\mu_i^S, W)^2 \le k \sum_{i=1}^k |S_i| \hat \sigma_{i, W}^2(S).
\]
\end{theorem}
\begin{proof}
We use the following lemma which is easy to verify.
\begin{lemma}\label{lem:pythagorean}
For points $p, p_1, \cdots, p_K \in \mathbb R^n$, if $\mu_{P} = \frac1K\sum_{i=1}^K p_i$, then
\[
\sum_{i=1}^K |p_i - p|^2 = K|p - \mu_P|^2 + \sum_{i=1}^K |p_i - \mu_P|^2.
\]
\end{lemma}

Let $M$ be the span of $\mu_1^S, \cdots, \mu_k^S$. For any $x\in \mathbb R^n$, denote by $\pi_M(x)$ the projection of $x$ onto $M$ and by $\pi_W(x)$ the projection of $x$ onto $W$. For any $i$, since $\mu_i^S \in M$, from Lemma \ref{lem:pythagorean} we have
\begin{equation*}
\begin{aligned}
\sum_{x\in S_i} |\pi_M(x)|^2 = |S_i|\cdot|\mu_i^S|^2 + \sum_{x\in S_i} |\pi_M(x) - \mu_i^S|^2 
\ge |S_i|\cdot|\mu_i^S|^2.
\end{aligned}
\end{equation*}
Taking the sum for $i=1, \cdots, k$, we have
\begin{equation} \label{eqn1-svd-thm}
\sum_{x\in S}|\pi_M(x)|^2 = \sum_{i=1}^k \sum_{x\in S_i} |\pi_M(x)|^2
\ge \sum_{i=1}^k |S_i|\cdot|\mu_i^S|^2
= \sum_{i=1}^k |S_i| \left( |\pi_W(\mu_i^S)|^2 + d(\mu_i^S, W)^2 \right).
\end{equation}

Let $e_1, \cdots, e_k$ be an orthonormal basis for $W$. For any $i$, using Lemma \ref{lem:pythagorean} we have
\begin{equation*}
\begin{aligned}
\sum_{x\in S_i} |\pi_W(x)|^2 &= |S_i|\cdot|\pi_W(\mu_i^S)|^2 + \sum_{x\in S_i} |\pi_W(x - \mu_i^S)|^2\\
&= |S_i|\cdot|\pi_W(\mu_i^S)|^2 + \sum_{j=1}^k \sum_{x\in S_i} |\pi_W(x - \mu_i^S)\cdot e_j|^2\\
&\le |S_i|\cdot|\pi_W(\mu_i^S)|^2 + k|S_i| \hat \sigma_{i, W}^2(S),
\end{aligned}
\end{equation*}
where the last inequality is because the variance of $S_i$ along any direction in $W$ is at most $\hat \sigma_{i, W}^2(S)$ (by definition of $\hat \sigma_{i, W}^2(S)$). Taking the sum for $i=1,\cdots,k$, we get
\begin{equation} \label{eqn2-svd-thm}
\sum_{x\in S} |\pi_W(x)|^2 \le \sum_{i=1}^k |S_i|\cdot|\pi_W(\mu_i^S)|^2 + k\sum_{i=1}^k |S_i| \hat \sigma_{i, W}^2(S)
\end{equation}

It is well-known that the SVD subspace, among all subspaces of dimension at most $k$, minimizes the sum of squared distances from points in $S$ to the subspace. Equivalently, it maximizes the sum of squared lengths of projections. Hence $\sum_{x\in S}|\pi_M(x)|^2 \le \sum_{x\in S}|\pi_W(x)|^2$. Then by comparing the RHSs of \eqref{eqn1-svd-thm} and \eqref{eqn2-svd-thm}, the proof is completed.
\end{proof}

Theorem \ref{thm:spectral-svd} essentially gives a way to lower bound the distances between component means after projection, in an average sense. If the means are well separated, at least some of them will continue to be separated after projection.


\subsection{Algorithm by spectral projection}

Now we describe the algorithm for learning a mixture of log-concave distributions.
The idea is to project the samples onto the SVD subspace and to classify samples in that subspace. However, since Theorem \ref{thm:spectral-svd} only indicates that the intermean distances are preserved in an average sense, we should not expect to classify all samples after one projection.
To overcome this, the algorithm runs in $k$ iterations, with the guarantee that in each iteration there exists one ``large" component that is well separated from others. The algorithm identifies this component, deletes the samples coming from it, and continues to the next iteration.


The inputs consist of $N$ i.i.d. samples, a weight lower bound $0<\varepsilon<1$, an error probability bound $0<\delta<1$, and a parameter $N_0<N$. For simplicity, we only describe the first iteration of the algorithm: (the rest iterations are essentially the same)
\begin{enumerate}
\item Choose a subset of samples $S$ of size $N_0$. Find the $k$-dimensional SVD subspace $W$ of $S$.
\item Delete $S$ and project the remaining samples, $T$, to the subspace $W$.
\item For each projected point $p$:
\begin{itemize}
\item Find its closest $\varepsilon N/2$ points. Let the set of these points be $T(p)$ and their mean be $\mu(p)$.
\item Let $A(p)$ be the matrix whose rows are $x-\mu(p)$ for all $x\in T(p)$. Compute the largest singular value $\sigma(p)$ of $A(p)$. (Note that $\sigma(p)^2$ is the maximum variance of $T(p)$ along any direction in $W$.)
\end{itemize}
\item Find a point $p_0$ which maximizes $\sigma(p_0)$. Let $T_0$ be the set of all points in $T$ whose projections are within distance $\frac{256\sqrt k \log(Nk/\delta)}{\varepsilon} \sigma(p_0)$ from $p_0$.
\item Identify $T_0$ as a component and delete it from samples.
\end{enumerate}

It is shown in \cite{Kannan08spectral} that this iterative spectral projection algorithm can correctly classify $N-kN_0$ samples with probability at least $1-\delta$ if (i) $N_0$ is large enough (polynomial in $n$, $\log k$, $\frac1\varepsilon$ and $\log\frac1\delta$); (ii) $N>C\frac{kN_0}{\varepsilon}$ for some universal constant $C$; (iii) the means of the components are separated as $|\mu_i - \mu_j| \ge 2^{11} (\sigma_i+\sigma_j)\cdot \frac{k^{3/2}}{\varepsilon^2} \cdot \log^2\frac{Nk}{\delta}$. Note that $kN_0$ samples are used for computing SVD subspaces and are not classified.

\subsection{Sketch of the analysis}

We give a sketch of the analysis of the algorithm.\\

\textbf{Sample properties.}
For a log-concave distribution, the distance from a random point to its mean has an exponentially decreasing distribution, so a sufficiently large number of i.i.d. samples will have good concentration properties. For the algorithm, it is shown that if $T$ is a set of at least $N_0$ samples, then w.h.p. for each $i$,
\begin{enumerate}[(a)]
\item $w_i - \frac{\varepsilon}{4} \le \frac{|T_i|}{|T|} \le w_i + \frac{\varepsilon}{4}$.
\item $|\mu_i - \mu_i^T| \le \frac{\sigma_i}{4}$.
\item For any subspace $W$, $\frac78 \sigma_{i, W}^2 \le \hat \sigma_{i, W}^2(T) \le \frac87 \sigma_{i, W}^2$.
\end{enumerate}
Here (a) directly follows from Chernoff bound, which ensures that the number of samples from each component is neither too small nor too large. (b) and (c) state that for each component the mean and variances of samples are close to their true values; they rely on the assumption that all the components are log-concave.
In the entire analysis, it is assumed that these properties always hold for the samples.\\

\textbf{Variation of Theorem \ref{thm:spectral-svd}.}
As a technical issue regarding independence, the algorithm has to use a set $S$ of samples to calculate the SVD subspace $W$ and project other samples $T$ onto this subspace, so Theorem \ref{thm:spectral-svd} cannot be applied directly.
However, given the sample properties (a)-(c), an alternative bound can be proved based on Theorem \ref{thm:spectral-svd}:
\begin{equation} \label{eqn:thm1-variation}
\sum_{i=1}^k |T_i| d(\mu_i^S, W)^2 \le 2k \sum_{i=1}^k |T_i| \hat \sigma_{i, W}^2(T).
\end{equation}


\textbf{Large components.}
The key step is to show that w.h.p. the algorithm can correctly identify one \emph{large component} in each iteration. A component $F_r$ is large if it satisfies $|T_r| \hat \sigma_{r, W}^2(T) \ge \beta \max_i|T_i| \hat \sigma_{i, W}^2(T)$, where $\beta = \frac{\varepsilon^3}{2^{14}k \log^2(\frac{Nk}{\delta})}$. Then it can be shown from \eqref{eqn:thm1-variation} and the sample properties that after projected onto $W$, the mean of a large component remains separated from the means of all other components.
Moreover, due to the light-tailed nature of log-concave distributions, w.h.p. every projected sample from any component $F_i$ lies within a ball with small radius centered at the projected mean of $F_i$. Intuitively, these facts explain why all projected samples from a large component $T_r$ are separated from those from other components. The authors prove that the point $p_0$ that maximizes $\sigma(p)$ must come from a large component $F_r$ and that if it does, the output will be the set $T_r$ exactly. This concludes the proof of correctness of the algorithm.





\section{Heavy-Tailed Distributions}

The paper \cite{Dasgupta05heavy-tail} focuses on heavy-tailed distributions when some of the  moments can even be infinite. In these cases, medians can serve as a more robust estimator than means or variances. Motivated by this, the median radius of a one-dimensional distribution is defined. 

\begin{definition}
Let $X$ be a random variable with cumulative distribution function $F(x)$. The center of $X$ is the minimum $c$ such that $F(c) = \frac{1}{2}$. The radius $R$ of $X$ is the smallest $R$ such that half of $X$'s density is in the interval $[c-R, c+R]$. This definition is generalized to multidimensional distributions by considering the centers coordinate-wise. 
\end{definition}

Let $\mathcal F_0$ be the class of distributions in $\mathbb{R}^n$ with independent coordinates, radius at most $R$ and symmetric and monotonically decreasing tails. Let $\mathcal F_1$ be a subset of $\mathcal F_0$ with one additional constraint: for any distribution $D \in \mathcal F_1$ centered at ${\mu}$ and any coordinate $D_i$ of $D$, a random sample $x$ from $D_i$ satisfies 
$$ \forall \alpha \geq 1, \Pr{(|x-\mu_i| \geq \alpha R)} \leq \frac{1}{2\alpha R}$$
where $\mu_i$ is the center of $D_i$.
This last condition is rather tame, since it is satisfied by any distribution with finite variance as well as other families.

The authors show that for a mixture of $k$ distributions $D_1, \cdots, D_k$ from $\mathcal F_0$ with centers at $\mu_1, \cdots, \mu_k$ such that $\| \mu_i - \mu_j \|_2 \geq \Omega\left(R \sqrt \frac k \epsilon \right)$ and $\frac{\|\mu_i - \mu_j\|_2}{\|\mu_i-\mu_j\|_{\infty}} \geq \Omega{\left( \sqrt{\frac{k}{\epsilon}}\right)}$ there is an algorithm that correctly classifies all but $\epsilon$ fraction of samples with high probability and uses a number of samples polynomial in $n, k, 1/\epsilon$ and $1/w_{\min}$. For mixtures from $\mathcal F_1$ a larger separation of the centers is required, but the slope condition is dropped. In particular, they require $\|\mu_i - \mu_j \|_2 \geq \Omega^* \left(\frac{Rk^{\frac{5}{2}}}{\epsilon^{2}}\right)$ in general and $\|\mu_i - \mu_j \|_2 \geq \Omega \left(\frac{Rk^{2}}{\epsilon^{2}}\right)$ for the easier case where centers are known.

The key insight of this algorithm is that the most elementary approach works: when centers are known, simply classify each point to the cluster whose center is nearest to it. The question in this case is how do we define ``near". The surprising answer is that the $\ell_1$ norm is sufficient. In fact, the authors show that there exist distributions for which the same algorithm and the $\ell_2$ norm misclassify a constant fraction of the points. A key difference between these norms is that $\ell_2$ is rotationally symmetric whereas $\ell_1$ is not. This makes clear an implicit assumption of this model: the distributions we are considering are coordinate-wise independent. If they were spherically symmetric instead, both norms would produce the same results; the paper does not solve the problem in such case. 

\subsection{Main technical lemmas}

The main technical lemma for learning mixtures with known centers comes from a robust property of arbitrary symmetric distributions. The condition basically says that a sample from the distribution is likely to be closer to the center of the distribution than the any fixed point sufficiently far from the center. The lemma further imposes some slope conditions on this fixed point, but the authors bypass this later for distributions from $\mathcal F_1$. 

\begin{lemma} \label{lem:heavy-tail-lem1}
Let $\epsilon$ and $C$ be constants and let $D$ be a distribution in $\mathbb R^n$ centered at the origin with radius $R$. Let $\mu$ be a point such that $\|\mu\|_2 \geq 4R(C+\frac{1}{\sqrt{\epsilon}})$, and having a slope ratio $\frac{\|\mu\|_2}{\|\mu\|_\infty} \geq 4(C+\frac{1}{\sqrt{\epsilon}})$. A point $x$ sampled from $D$ will satisfy 
\begin{equation*}
\|x-\mu\|_1 - \|x\|_1 \geq C\|\mu\|_2 \geq C^2R 
\end{equation*}
with probability at least $1-\epsilon$. 
 
\end{lemma} 

\begin{lemma} \label{lem:heavy-tail-lem2}
Fix $\epsilon \leq \frac{1}{10}$. Suppose $D_1 \in \mathcal F_1$ and $\mu \in \mathbb{R}^n$ satisfies $\|\mu\|_2 \geq \frac{6000R}{\epsilon^2}$. Then $x$ sampled from $D_1$ will satisfy 
$$\|x - \mu\|_1 - \|x\|_1 \geq \frac{\|\mu\|_2}{15}$$
with probability at least $1-\epsilon$. 
\end{lemma}

The proof of the Lemma \ref{lem:heavy-tail-lem1} results from an application of Chebyshev's inequality with appropriate lower bounds on the expectation and upper bounds on the variance. The proof of Lemma \ref{lem:heavy-tail-lem2} separates large and small coordinates by absolute values of $\mu_i$ with a threshold at $O(R/\epsilon)$. For the larger group, the additional restriction of $\mathcal F_1$ is used while the previous lemma is still valid for the smaller groups. Both results put together with union bound %some common probabilistic tools
provide the following theorem. 

\begin{theorem}
Consider a mixture of $k$ distributions $D_1,\cdots,D_k$ with known centers $\mu_1,\cdots,\mu_k$. If either of the following conditions is met, then classification according to nearest center in the $\ell_1$ norm succeeds with probability at least $1-\epsilon$.

\begin{itemize}
	\item For every $i\not=j$, $\|\mu_i - \mu_j\|_2 \geq \Omega\left( R \sqrt{\frac{k}{\epsilon}}\right)$ and $\frac{\|\mu_i - \mu_j\|_2}{\|\mu_i - \mu_j\|_\infty} \geq \Omega \left( \sqrt{\frac{k}{\epsilon}}\right)$ or
	\item Each distribution belongs to the class $\mathcal F_1$ and for every $i\not= j$, $\|\mu_i - \mu_j\|_2 \geq \Omega\left(R\frac{k^2}{\epsilon^2}\right)$.
	
\end{itemize}
\end{theorem}

\subsection{Algorithm for the general case}

The intuition for the general case where the centers are not known is the following. If we knew the centers, then we could know that the $\ell_1$ norm would be sufficient to classify the points. Consider partitioning the coordinates into two groups and clustering the points independently using both partitions, then we should get approximately the same clusters. 

The algorithm iteratively selects a set of sample points $S_0$ to build the clusters and another group $S_1$ to cross-validate the clusters. All possible clusterings of $S_0$ into $k+1$ groups are considered. The algorithm then computes the median of each cluster except the last one for both partitions and then uses these centers to classify the points of $S_1$. If the first and second clusters are similar for each cluster and the $(k+1)$-st cluster is small, the algorithm accepts. Otherwise, it will pick another partition of the coordinates and try again.  

At first, it might seem discouraging that the algorithm might have to go through every possible partition of the coordinates, but the authors quickly show that most partitions will be good provided that the centers are sufficiently not axis-aligned. 





\bibliography{18-409-FP.bib} 
\bibliographystyle{plain} 

\end{document}
	% line of code telling latex that your document is ending. If you leave this out, you'll get an error
